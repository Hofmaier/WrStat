\documentclass[ngerman, a4paper, 10pt, twocolumn, DIV20, headings=small]{scrartcl}
\usepackage[german]{babel}
\usepackage[utf8]{inputenc}
\usepackage{amsmath}
\usepackage{amsthm}
\usepackage{amssymb}
\usepackage{amsfonts}

\newtheorem{definition}{Definition}
\newtheorem{satz}{Satz}

\begin{document}

\section{Kombinatorik}
\label{sec:kombinatorik}

\begin{tabular}{l l}
  Produktregeln &  \\
  $k$-Permutation &  $P(n,k) = \frac{n!}{(n-k)!}$\\
  Permutation & $P(n,n) = n!$ \\
  Kombination & $ C^n_k=\binom{n}{k} $ \\ 
\end{tabular}

\section{Ereignisse}
\label{sec:ereignisse}

\begin{definition}
  Alle m"oglichen sich gegenseitig ausschliessenden Ergebnisse eines Zufallexperiments bilden die Menge aller Elementarereignisse $\Omega$.
\end{definition}

\begin{definition}
  Ein Ereignis $A$ ist eine Teilmenge von $\Omega$. $A \subset \Omega$. Es ist eingetreten wenn das Ergebnis des Zufallexperiment ein Element von $A$ ist. Einelementige Teilmengen von $\Omega$ sind Elementarereignisse. Die Potenzmenge ${\cal P}(\Omega)$ beinhaltet alle theoretisch m"oglichen Ereignisse.
\end{definition}

\begin{tabular}{|l|c|}
\hline
Begriff&Modell\\
\hline
Elementarereignis&$\omega$\\
alle Elementarereignisse&$\Omega$\\
Ereignis&$A\subset\Omega$\\
sicheres Ereignis&$\Omega$\\
unm"ogliches Ereignis&$\emptyset$\\
$A$ und $B$&$A\cap B$\\
$A$ oder $B$&$A\cup B$\\
$A$ hat $B$ zur Folge, $A\Rightarrow B$&$A\subset B$\\
nicht $A$&$\Omega\setminus A$\\
\hline
\end{tabular}

\section{Wahrscheinlichkeit}
\label{sec:wahrscheinlichkeit}

\begin{satz}
  Bei einem Laplace-Experiment $| \Omega |$ m"oglichen Ergebnissen gilt
\[
P(A) = \frac{|A|}{|\Omega|}
\]
\end{satz}

\begin{satz}
F"ur zwei Ereignisse $A$ und $B$ ist die Wahrscheinlichkeit, dass das Ereignis $A$ oder $B$ eintritt gleich
\[
P(A\cup B) = P(A) + P(B) - P(A\cap B)
\]
falls $ A \cup B = {} $
\[
P(A\cup B) = P(A) + P(B)
\]
\end{satz}

\begin{definition}
Die Ereignisse $A$ und $B$ heissen {\bf unabh"angig}, wenn gilt:
\[
P(A\cap B) = P(A)\cdot P(B).
\]
\end{definition}

\begin{satz}
Ist $B_i$ eine Folge paarweise disjunkter Mengen mit $\bigcup_{i=0}^{n}B_i=\Omega$, dann gilt f"ur jedes Ereignis $A$
\[
P(A)=\sum_{i=0}^{n}P(A|B_i)\cdot P(B_i).
\]
\end{satz}

\begin{satz}[Satz von Bayes]
F"ur zwei Ereignisse $A$ und $B$ mit $P(B)\ne0$ gilt
\[
P(A|B)=\frac{P(B|A)\cdot P(A)}{P(B)}.
\]
\end{satz}

\section{Erwartungswert}
\label{sec:erwartungswert}

\[
E=\sum_{i=1}^{n}g_iP(A_i).
\]

\begin{satz}
\label{rechenregeln-erwartungswert}
Sind $X$ und $Y$ Zufallsvariable mit Werten in $\mathbb{R}$ ,
und $\lambda\in\mathbb{R}$, dann gilt
\begin{enumerate}
\item $E(X+Y)=E(X)+E(Y)$
\item $E(\lambda X)=\lambda E(X)$

\end{enumerate}
\end{satz}

\section{Varianz}
\label{sec:varianz}

\begin{definition}
Sei $X\colon\Omega\to\mathbb{R}$ eine Zufallsvariable, dann
heisst die durch 
\[
\operatorname{var}(X)=E((X-E(X))^2)
\]
 definierte Gr"osse $\operatorname{var}(X)$ die
{\bf Varianz} von $X$. Es ist insbesondere
\[
\operatorname{var}(X)=E((X-E(X))^2)=E(X^2)-E(X)^2
\]
\end{definition}

\begin{satz}
\label{rechenregeln-varianz}
Seien $X$ und $Y$ unabh"angige Zufallsvariable, dann haben
Summe und Produkt folgende Varianz:
\begin{eqnarray*}
\operatorname{var}(\lambda X)&=&\lambda^2\operatorname{var}(X)\\
\operatorname{var}(X+Y)&=&\operatorname{var}(X)+\operatorname{var}(Y)\\
\operatorname{var}(XY)&=&\operatorname{var}(X)\operatorname{var}(Y)
+
\operatorname{var}(Y)E(X)^2+\operatorname{var}(X)E(Y)^2
\end{eqnarray*}
\end{satz}

\section{Lineare Regression}
\label{sec:lineareregression}

Gegeben seien Wertepaare $(x_1, y_1), (x_2, y_2), ...$. Gesucht ist eine Funktion, die die Abh"angigkeit der $y_i$-Werte von den $x_i$-Werten m"oglichst gut beschreibt. Ist diese Funktion die Gerade gilt:
\begin{satz}
  Die Gerade mit der Gleichung $y = ax + b$, minimiert die Varianz
\[
var(aX + b - Y)
\]
wenn

\begin{eqnarray*}
a&=&\frac{E(XY)-E(X)E(Y)}{(E(X^2)-E(X)^2)}=\frac{\operatorname{cov}(X,Y)}{\operatorname{var}(X)}\\
b&=&E(Y)-E(X)a
\end{eqnarray*}

\end{satz}

\section{R}
\label{sec:r}

V

\subsection{Funktionen}
\label{sec:r-funktionen}

\begin{tabular}{l l}
  Fakult"at & \verb|factorial()| \\
  Binomialkoeffizient & \verb|choose(n,k)| \\
\end{tabular}

\subsection{Vektoren}
\label{sec:r-vektoren}

\begin{verbatim}
x <- c(6,10,14)
\end{verbatim}
erzeugt einen Vektor mit den Elementen $6, 10, 14$ und den Vektor der Variablen \verb|x| zu.

\section{Octave}
\label{sec:octave}

\begin{tabular}{l l}
  Fakult"at & \verb|factorial()| \\
  Kombinationen & \verb|nchoosek(n,k)| \\
\end{tabular}

\end{document}

%%% Local Variables: 
%%% mode: latex
%%% TeX-master: t
%%% End: 
