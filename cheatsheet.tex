\documentclass[ngerman, a4paper, 10pt, twocolumn, DIV20, headings=small]{scrartcl}
\usepackage[german]{babel}
\usepackage[utf8]{inputenc}
\usepackage{amsmath}
\usepackage{amsthm}

\newtheorem{definition}{Definition}
\newtheorem{satz}{Satz}

\begin{document}

\section{Kombinatorik}
\label{sec:kombinatorik}

\begin{tabular}{l l}
  Produktregeln &  \\
  $k$-Permutation &  $P(n,k) = \frac{n!}{(n-k)!}$\\
  Permutation & $P(n,n) = n!$ \\
  Kombination & \\
\end{tabular}

\section{Ereignisse}
\label{sec:ereignisse}

\begin{definition}
  Alle m"oglichen sich gegenseitig ausschliessenden Ergebnisse eines Zufallexperiments bilden die Menge aller Elementarereignisse $\Omega$.
\end{definition}

\begin{definition}
  Ein Ereignis $A$ ist eine Teilmenge von $\Omega$. $A \subset \Omega$. Es ist eingetreten wenn das Ergebnis des Zufallexperiment ein Element von $A$ ist. Einelementige Teilmengen von $\Omega$ sind Elementarereignisse. Die Potenzmenge ${\cal P}(\Omega)$ beinhaltet alle theoretisch m"oglichen Ereignisse.
\end{definition}

\begin{tabular}{|l|c|}
\hline
Begriff&Modell\\
\hline
Elementarereignis&$\omega$\\
alle Elementarereignisse&$\Omega$\\
Ereignis&$A\subset\Omega$\\
sicheres Ereignis&$\Omega$\\
unm"ogliches Ereignis&$\emptyset$\\
$A$ und $B$&$A\cap B$\\
$A$ oder $B$&$A\cup B$\\
$A$ hat $B$ zur Folge, $A\Rightarrow B$&$A\subset B$\\
nicht $A$&$\Omega\setminus A$\\
\hline
\end{tabular}

\section{Wahrscheinlichkeit}
\label{sec:wahrscheinlichkeit}

\begin{satz}
  Bei einem Laplace-Experiment $| \Omega |$ m"oglichen Ergebnissen gilt
\[
P(A) = \frac{|A|}{|\Omega|}
\]
\end{satz}

\begin{definition}
Die Ereignisse $A$ und $B$ heissen {\bf unabh"angig}, wenn gilt:
\[
P(A\cap B) = P(A)\cdot P(B).
\]
\end{definition}

\begin{satz}
Ist $B_i$ eine Folge paarweise disjunkter Mengen mit $\bigcup_{i=0}^{n}B_i=\Omega$, dann gilt f"ur jedes Ereignis $A$
\[
P(A)=\sum_{i=0}^{n}P(A|B_i)\cdot P(B_i).
\]
\end{satz}

\section{Octave}
\label{sec:octave}

\begin{tabular}{l l}
  Fakult"at & \verb|factorial()| \\
  Kombinationen & \verb|nchoosek(n,k)| \\
\end{tabular}

\end{document}

%%% Local Variables: 
%%% mode: latex
%%% TeX-master: t
%%% End: 
